\documentclass[a4paper,12pt]{article}

\usepackage[utf8]{inputenc}
\usepackage[T1]{fontenc}
\usepackage{amsfonts}
\usepackage{amsmath}
\usepackage{amsthm}
\usepackage{algorithm}
\usepackage{algpseudocode}

\newtheorem{proposition}{Proposition}
\newtheorem{theorem}{Theorem}

\newcommand{\Z}{\mathbb{Z}}

\DeclareMathOperator{\shake}{SHAKE256}
\DeclareMathOperator{\asn}{ASN1}
\DeclareMathOperator{\der}{DER}
\DeclareMathOperator{\derasn}{DER-ASN1}
\DeclareMathOperator{\sha}{SHA3}

\title{Hashing into quadratic residues \\ modulo a safe prime composite}
\author{Sietse Ringers}
\date{February 25, 2021}

\begin{document}

\maketitle

\noindent Set $n = pq$, and set $p'$, $q'$ such that $p = 2p' + 1$ and $q = 2q' + 1$. Suppose $p$, $q$ are safe primes, i.e., $p'$ and $q'$ are also prime. Call a function $H$ a cryptographic hash function if it is (second) pre-image resistant and collision resistant. Given a cryptographic hash function $H$ whose output is sufficiently long, it is possible to define another hash function $H_n$ as the composition of $H$ and squaring modulo $n$. This document proves in section 1 that then $H_n$ is also a cryptographic hash function, after first showing three preliminary propositions that we use in our proof. Then in section 2 we provide an explicit description of the hash function $H_n$ in pseudocode.

We write $\Z^*_n$ for $(\Z/n\Z)^*$, the multiplicative group of the integers modulo $n$ having an inverse (i.e. $0 < x < n$ with $\gcd(x, n) = 1$). Additionally we write $QR_n = (\Z^*_n)^2 = \{ x^2 \mid x \in \Z^*_n \}$ for the group of quadratic residues modulo $n$.

\section{Proving security}

\begin{proposition}
$\Z^*_n$ contains exactly four square roots of 1, i.e. elements $X$ such that $X^2 = 1$, namely:
\begin{itemize}
  \item $1 \bmod n$
  \item $n - 1 \bmod n$
  \item $R := Pp - Qq \bmod n$, where $P$, $Q$ are the integers such that $Pp + Qq = 1$, given by the extended Euclidean algorithm
  \item $n - R \bmod n$
\end{itemize}
\end{proposition}

\begin{proof}
It is clear that $1$ and $n - 1$ square to $1 \bmod n$. Due to the Chinese Remainder Theorem (CRT), $\Z^*_n$ is isomorphic to $\Z^*_p \times \Z^*_q$. The order of the two group factors is $p-1 = 2p'$ and $q-1 = 2q'$ respectively. Both of those groups have a subgroup of order $2$, namely the ones generated by $-1 \bmod p$ and $-1 \bmod q$ respectively, and because of Lagrange's theorem those must be the only such subgroups. Therefore because of the CRT isomorphism, $\Z^*_n$ has two distinct subgroups of order two, generated by $(1, -1)$ and $(-1, 1)$. Under the inverse of that isomorphism these are $R = Pp - Qq \bmod n$ and $-R = n - R \bmod n$.
\end{proof}

\begin{proposition}
Any quadratic residue $Y = X^2 \bmod n \in QR_n$ unequal to $1$ has exactly four square roots in $\Z^*_n$, namely $X \bmod n$, $n-X \bmod n$, $RX \bmod n$ and $n-RX \bmod n$. Two of these have representatives smaller than or equal to $(n-1)/2$.
\end{proposition}

\begin{proof}
That the mentioned numbers square to $Y$ is easily seen using direct computation. Furthermore, if $X$ and $Z$ have the same square ${}\bmod n$, then $X/Z \bmod n$ squares to $1 \bmod n$, so if any $Y$ had more than four distinct roots then this would yield a fifth square root of one which does not exist.

As to the second claim of the proposition, one of $X$ and $n-X$ must be smaller than or equal to $(n-1)/2$, and then the other will be larger. The same must hold of the smallest representatives of $RX \bmod n$ and $n-RX \bmod n$.
\end{proof}

\begin{proposition}
If one knows one of the nontrivial square roots of $1 \in QR_n$ (i.e. not $1$ or $n-1$), then one can factor $n$.
\end{proposition}

\begin{proof}
Denote the square root again with $R$. Since $R^2 = 1 \bmod n$ we have $R^2 - 1 = (R+1)(R-1) = 0 \bmod n$; i.e. for some integer $a$, $(R+1)(R-1)$ is of the form $(R+1)(R-1) = an = apq$, with $a \neq 0$ since $R \neq 1$. Now since $p$ is prime, it must divide one of the two factors, $R+1$ or $R-1$. Since $R+1 \neq pq = n$ (as we assumed the square root was nontrivial), it follows that $q$ must divide the other factor. So the factors of $n$ are $\gcd(n, R-1)$ and $\gcd(n, R+1)$.
\end{proof}

\noindent Since $p = 2p'+1$ and $q = 2q'+1$ are safe primes, the order of $\Z^*_n$ is $\phi(n) = (p-1)(q-1) = 4p'q'$. Then it is easy to see that the order or $QR_n$ equals $p'q'$. For example, using CRT, the fact that $\Z^*_p$ and $\Z^*_q$ are cyclic, and then CRT again, we have
\[
  \Z^*_n \cong \Z^*_p \times \Z^*_q \cong \Z_{2p'} \times \Z_{2q'} \cong \Z_2 \times \Z_2 \times \Z_{p'} \times \Z_{q'}
\]
Then the quadratic residues in $\Z^*_n$ are those whose two components in the two group factors $\Z_2$ equal $0$. This means that the order of $QR_n$ is indeed $p'q'$.

\begin{theorem}
If $f: \{0,1\}^* \to QR_n$ is any function such that $f(x) = 1$ does not happen or happens with negligible probability, and if factoring is hard, then its output will be a generator of $QR_n$ with overwhelming probability.
\end{theorem}

\begin{proof}
Suppose $G = f(x)$ is not a generator; that is, its order is not the maximal order $p'q'$. Without loss of generality let its order be $p'$, so that $1 = G^{p'} \bmod n$. Since $n = pq$, reducing modulo $p$ gives the identity 
\[ 1 = G^{p'} \bmod q = (G \bmod q)^{p'} \bmod q. \]
Now $G \bmod q$ is an element of $\Z^*_q$, whose group order is $2q'$, and since $G$ is a quadratic residue the order of $G \bmod q$ cannot be $2q'$, so it must be either $1$ or $q'$. In the latter case, our identity $(G \bmod q)^{p'} = 1 \bmod q$ would imply that $q'$ divides $p'$ which is impossible because $p'$ is prime. Therefore, the order of $G \bmod q$ in $\Z^*_q$ is $1$, i.e. $G = 1 \bmod q$. This implies that $G - 1 = aq$ for some $a$, i.e., $\gcd(n, G-1) = q$.
\end{proof}

\begin{theorem}
Let $H: \{0,1\}^* \to [2, (n-1)/2]$ be a cryptographic hash function (i.e. it is collision resistant and (second) pre-image resistant). Define $H_n : \{0,1\}^* \to QR_n$ by $H_n(x) = H(x)^2 \bmod n$. If factoring is hard, then $H_n$ is also a cryptographic hash function, which outputs generators of $QR_n$ with overwhelming probability.
\end{theorem}

\begin{proof}
First note that the output of $H$ will have with overwhelming probability a multiplicative inverse ${}\bmod n$, i.e. $\gcd(H(x), n) = 1$, because if not, then $\gcd(H(x), n)$ will factor $n$. So with some abuse of notation, we may consider the range of $H$ to be a subset of $\Z^*_n$, so that the range of $H_n$ is indeed $QR_n$.

Suppose $H_n$ is not collision resistant, so let $x_1 \neq x_2$ be such that $H_n(x_1) = H_n(x_2) \bmod n$. Then by Proposition 2, $H(x_1)$ equals $H(x_2)$ or $n - H(x_2)$ or $RH(x_2) \bmod n$ or $n - RH(x_2) \bmod n$. It cannot be $n - H(x_2)$ since that exceeds $(n-1)/2$. Similarly, of $RH(x_2) \bmod n$ and $n - RH(x_2) \bmod n$, only one will have a smallest representative that is smaller than $(n-1)/2$. Suppose without loss of generality that it is $RH(x_2)$. Summarizing, then, we have either $H(x_1) = H(x_2)$ or $H(x_1) = RH(x_2)$. Now in the latter case we have $H(x_1)/H(x_2) = R \bmod n$: one of the nontrivial square roots of 1 (since $1 < H(\cdot) < n-1$). So if the latter case holds with non-negligible probability, then we have a non-negligible chance of being able to factor $n$, by Proposition 3. Thus we must have $H(x_1) = H(x_2)$. So any algorithm that breaks collision resistance of $H_n$ can be used to break that of $H$, which is impossible since we assumed $H$ to be collision resistance.

Collision resistance implies second pre-image resistance. Ordinary pre-image resistance can be proven with an almost identical argument as above.

The fact that $H_n$ outputs generators with overwhelming probability is proven in the previous theorem.
\end{proof}

\noindent Because the hash function is the composition of $H$ and squaring modulo $n$, and because for each integer smaller than or equal to $(n-1)/2$ there is exactly one other such integer that squares to the same quadratic residue by Proposition 2, $H_n$ has exactly twice as much collisions as $H$ itself. This is to be expected, however, since the range of $H_n$ is half as large as the maximal range of $H$ (which we take to be the lower half of $\Z^*_n$, as above). Additionally, the fact that all quadratic residues have exactly two roots smaller than the upper bound ensures that this non-injectiveness of the square function does not cause particular values of $QR_n$ to be returned by $H_n$ more often than others. Summarizing, the output of $H_n$ ``appears as random'' as can be expected.

\section{Instantiation and implementation}
In this section we describe the cryptographic hash function $H_n$ in more detail. For convenience, we interpret the output of our hash functions as large integers; that is, we assume an implicit conversion of the output bytes to integers.

Generally the construction below can be done for any hash function $H$ of sufficiently long output length, but for concreteness we take $H(x) = \shake(x, d)$. Here $\shake$ from the $\sha$ function family is a so-called Extendible Output Function (XOF): a function that has variable output length, specified in bits as the second parameter $d$, with the property that for any fixed $d$ the function $\shake(\cdot, d)$ is a cryptographic hash function, and moreover if $d' > d$ then the first $d$ bits of $\shake(\cdot, d')$ coincide with $\shake(\cdot, d)$.

Let $L_n = |n|$ be the length in bits of the modulus (i.e. 1024, 2048 or 4096). As the theorem above states, for the security of $H_n$ it is important that the cryptographic hash function $H$ has the appropriate maximum output; specifically, its output should be smaller than or equal to $(n-1)/2$. Now since $\shake(\cdot, d)$ outputs $d$ bits the upper limit of its output is $2^d$ instead of $(n-1)/2$. Setting $d = L_n-1 = |(n-1)/2|$, our hash functions will thus sometimes output an integer smaller than $2^d$ but larger than $(n-1)/2$. We can ``fix'' that by prepending our input bytes with a counter $i$ starting at 0, i.e. when hashing $x$ we return $H(0 || x)$ if that is below the upper bound, and if it exceeds $(n-1)/2$ we increment $i$ until $H(i || x) \leq (n-1)/2$. We do the same in the (unlikely) case that $H$ outputs 0 or 1. To prevent attacks where $x$ is crafted with a specific $i$ as its first few bits, one should use an encoding such as $\derasn$ for $i || x$.
Note that this does not mean that an implementation has to include a generic $\asn$ parser; instead one can work out once and then hardcode the bytes of a $\der$ encoding of the following $\asn$ sequence:

\begin{verbatim}
HashInput ::= SEQUENCE {
  i INTEGER,
  x OCTET STRING
}
\end{verbatim}

\noindent Finally, in implementations it might only be possible to specify the output length of $\shake$ in bytes instead of in bits. In this case, one can simply request $L_n/8$ output bytes and then discard the rightmost bit to end up with the required $d = L_n-1$ bits.

A description of the algorithm computing $H_n$ summarizing the above may be found in pseudocode below. We assume there that $\shake$ takes its output length as the second parameter in bits.

\begin{algorithm}
  \caption{Cryptographic hash function $H_n : \{0,1\}^* \to QR_n$}
  \begin{algorithmic}
    \Function{$H_n$}{$x$}
      \State $i \gets 0$
      \Repeat
        \State $O \gets \shake(\derasn(i, x), L_n-1)$
        \State $i \gets i + 1$
      \Until{$1 < O \leq (n-1)/2$}
      \State \Return{$O^2 \bmod n$}
    \EndFunction
  \end{algorithmic}
\end{algorithm}

\end{document}
